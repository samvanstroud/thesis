\chapter{The Large Hadron Collider and the \ATLAS Detector}
\label{chap:detector}

\section{Overview}
The Large Hadron Collider (\LHC) at \CERN has extended the frontiers of particle physics through its unprecedented energy and luminosity.
In 2010, the \LHC collided proton bunches, each containing more than $10^{11}$ particles, 20 million times per second, providing \SI{7}{\TeV} proton-proton collisions at instantaneous luminosities of up to \peakLumi.


\subsection{The ATLAS Detector}\label{sec:atlas-detector}
%% from https://twiki.cern.ch/twiki/bin/view/AtlasProtected/PubComCommonText

% Footnote with ATLAS coordinate system
\newcommand{\AtlasCoordFootnote}{%
ATLAS uses a right-handed coordinate system with its origin at the nominal interaction point in the centre of the detector and the \(z\)-axis along the beam pipe. The \(x\)-axis points from the interaction point to the centre of the LHC ring, and the \(y\)-axis points upwards. Cylindrical coordinates \((r,\phi)\) are used in the transverse plane, \(\phi\) being the azimuthal angle around the \(z\)-axis. The pseudorapidity is defined in terms of the polar angle \(\theta\) as \(\eta = -\ln \tan(\theta/2)\). Angular distance is measured in units of \(\DeltaR \equiv \sqrt{(\Delta\eta)^{2} + (\Delta\phi)^{2}}\).}

The ATLAS detector at the LHC covers nearly the entire solid angle around the collision point.\footnote{\AtlasCoordFootnote}
It consists of an inner tracking detector surrounded by a thin superconducting solenoid, electromagnetic and hadron calorimeters,
and a muon spectrometer incorporating three large superconducting air-core toroidal magnets.

The inner-detector system (ID) is immersed in a \SI{2}{\tesla} axial magnetic field 
and provides charged-particle tracking in the range \(|\eta| < 2.5\).
The high-granularity silicon pixel detector covers the vertex region and typically provides four measurements per track, 
the first hit normally being in the insertable B-layer (IBL) installed before Run~2~\cite{ATLAS-TDR-19,PIX-2018-001}.
It is followed by the silicon microstrip tracker (SCT), which usually provides eight measurements per track.
These silicon detectors are complemented by the transition radiation tracker (TRT),
which enables radially extended track reconstruction up to \(|\eta| = 2.0\). 
The TRT also provides electron identification information 
based on the fraction of hits (typically 30 in total) above a higher energy-deposit threshold corresponding to transition radiation.
Reconstructed charged particles are assumed to have a charge of $\pm 1$.

A complete overview of the ATLAS detector is provided in Ref.~\cite{PERF-2007-01}.


%The calorimeter system covers the pseudorapidity range \(|\eta| < 4.9\).
%Within the region \(|\eta|< 3.2\), electromagnetic calorimetry is provided by barrel and 
%endcap high-granularity lead/liquid-argon (LAr) calorimeters,
%with an additional thin LAr presampler covering \(|\eta| < 1.8\)
%to correct for energy loss in material upstream of the calorimeters.
%Hadron calorimetry is provided by the steel/scintillator-tile calorimeter,
%segmented into three barrel structures within \(|\eta| < 1.7\), and two copper/LAr hadron endcap calorimeters.
%The solid angle coverage is completed with forward copper/LAr and tungsten/LAr calorimeter modules
%optimised for electromagnetic and hadronic energy measurements respectively.
%
%The muon spectrometer (MS) comprises separate trigger and
%high-precision tracking chambers measuring the deflection of muons in a magnetic field generated by the superconducting air-core toroidal magnets.
%The field integral of the toroids ranges between \num{2.0} and \SI{6.0}{\tesla\metre}
%across most of the detector. 
%Three layers of precision chambers, each consisting of layers of monitored drift tubes, covers the region \(|\eta| < 2.7\),
%complemented by cathode-strip chambers in the forward region, where the background is highest.
%The muon trigger system covers the range \(|\eta| < 2.4\) with resistive-plate chambers in the barrel, and thin-gap chambers in the endcap regions.
%
%Interesting events are selected by the first-level trigger system implemented in custom hardware,
%followed by selections made by algorithms implemented in software in the high-level trigger~\cite{TRIG-2016-01}. 
%The first-level trigger accepts events from the \SI{40}{\MHz} bunch crossings at a rate below \SI{100}{\kHz},
%which the high-level trigger further reduces in order to record events to disk at about \SI{1}{\kHz}.
%
%An extensive software suite~\cite{ATL-SOFT-PUB-2021-001} is used in the reconstruction and analysis of real
%and simulated data, in detector operations, and in the trigger and data acquisition systems of the experiment.

\section{Trigger system}
\label{sec:bg-theory:triggers}
An LHCb trigger table borrowed from \texttt{hepthesis} is shown in \TableRef{tab:bg-theory:trigger_details}:

\begin{table}[bht]
  \begin{tabular}{lllll}
                & L0              & L1              & HLT             \\
    \midrule
    Input rate  & \SI{40}{\MHz} & \SI{1}{\MHz}  & \SI{40}{\kHz} \\
    Output rate & \SI{1}{\MHz}  & \SI{40}{\kHz} & \SI{2}{\kHz}  \\
    Location    & On detector     & Counting room   & Counting room   \\
  \end{tabular}
  \caption{Characteristics of the trigger levels and offline analysis.}
  \label{tab:bg-theory:trigger_details}
\end{table}

\section{Reconstructed Physics Objects}\label{sec:physics-objects}

\subsection{Tracks}\label{sec:tracks}

The trajectories of charged particles are reconstructed as tracks from the energy depositions (hits) of the particles as they traverse the sensitive elements of the inner detector.
Track selection follows the loose selection described in Ref.~\cite{ATL-PHYS-PUB-2020-014} and outlined in~\cref{tab:track_selections}, which was found to improve the flavour tagging performance compared to previous tighter selections, whilst ensuring good resolution of tracks and a low fake rate~\cite{PERF-2015-08}.
The transverse IP $d_0$ and longitudinal IP $z_0$ are measured with respect to the hard scatter primary vertex, defined as the reconstructed primary vertex (PV) with the largest sum of the transverse momentum ($\pt$) of the associated tracks squared, $\sum \pt^2$.
%The reconstructed tracks are required to satisfy the quality requirements in \cref{tab:track_selections}.

\begin{table}[!htbp]
    \small
    \centering
    \caption{
        Quality selections applied to tracks,
        where $d_0$ is the transverse IP of the track, $z_0$ is the longitudinal IP with respect to the PV and $\theta$ is the track polar angle.
        Shared hits are hits used on multiple tracks which have not been classified as split by the cluster-splitting neural networks~\cite{PERF-2015-08}.
        Shared hits on pixel layers are given a weight of 1, while shared hits in the SCT are given a weight of 0.5.
        A hole is a missing hit, where one is expected, on a layer between two other hits on a track.
        }
    \begin{tabular}{ll}
        \toprule 
        \textbf{Parameter} & \textbf{Selection} \\
        \hline
        $\pt$                & $> 500$ MeV \\
        $|d_0|$              & $< 3.5$ mm \\
        $|z_0 \sin\theta|$   & $< 5$ mm \\
        Silicon hits         & $\ge 8$ \\
        Shared silicon hits  & $< 2$ \\
        Silicon holes        & $< 3$ \\
        Pixel holes          & $< 2$ \\
        \bottomrule
    \end{tabular}
    \vspace{4mm}
    \label{tab:track_selections}
\end{table}

\subsection{Jets}\label{sec:jets}

Jets are reconstructed from particle-flow objects \cite{PERF-2015-09} using the anti-$k_T$ algorithm \cite{Cacciari:2008gp} with a radius parameter of $0.4$.
The jet energy scale is calibrated according to Ref.~\cite{PERF-2016-04}.
Jets are also required not to overlap with a generator-level electron or muon from \Wboson boson decays.
All jets are required to have a pseudorapidity $|\eta| < 2.5$ and $\pt > \SI{20}{\GeV}$. 
Additionally, a standard selection using the Jet Vertex Tagger (JVT) algorithm at the tight working point is applied to jets with $\pt < \SI{60}{\GeV}$ and $|\eta| < 2.4$ in order to suppress pileup contamination \cite{ATLAS-CONF-2014-018}.
Tracks are associated to jets using a \DeltaR association cone, the width of which decreases as a function of jet \pt, with a maximum cone size of $\DeltaR \approx 0.45$ for jets with $\pt = \SI{20}{\GeV}$ and minimum cone size of $\DeltaR \approx 0.25$ for jets with $\pt > \SI{200}{\GeV}$. 
If a track is within the association cones of more than one jet, it is assigned to the jet which has a smaller $\DeltaR(\text{track}, \text{jet})$.

Jet flavour labels are assigned according to the presence of a truth hadron within ${\DeltaR(\text{hadron},\text{jet})<0.3}$ of the jet axis. If a \bhadron is found the jet is labelled a \bjet. In the absence of a \bhadron, if a \chadron is found the jet is called a \cjet.
If no \borchadrons are found, but a $\tau$ is found in the jet, it is labelled as a $\tau$-jet, else it is labelled as a \ljet.


\begin{itemize}
  \item Jet finding algorithms
\end{itemize}

\subsection{Leptons}\label{sec:leptons}