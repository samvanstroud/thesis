% no more than 300 words

This thesis presents investigations into the challenges of, and potential improvements to, \bjet identification (\btagging) at the \ATLAS experiment at the Large Hadron Collider (\LHC).
A primary application of \btagging is the study of the Higgs boson, which preferentially decays to a pair of \bquarks.
A focus is placed on the high transverse momentum regime, which is a critical region in which to study the Standard Model, but also a region within which successful \btagging becomes difficult.

As \btagging relies on the successful reconstruction of charged particle trajectories (tracks), the tracking performance is investigated and potential improvements are assessed.
Track reconstruction becomes increasingly difficult at high transverse momentum due to the increased multiplicity and collimation of tracks, and also due to the presence of displaced tracks from the decay of a long-flying \bhadron.
The investigations reveal that the quality selections applied during track reconstruction are suboptimal for \bhadron decay tracks inside high transverse momentum \bjets, motivating future studies into the optimisation of these selections.

Two novel approaches are developed to improve \btagging performance.
Firstly, an algorithm which is able to classify the origin of tracks is used to select a more optimal set of tracks for input to the \btagging algorithms.
Secondly, a graph neural network (GNN) jet flavour tagging algorithm has been developed.
This algorithm directly accepts jets and tracks as inputs, making a break from previous algorithms which relied on the outputs of intermediate taggers.
The model is trained to simultaneously predict the jet flavour, track origins, and the spatial track-pair compatibility, and demonstrates marked improvements in \btagging performance both at low and high transverse momenta.
The closely related task of \cjet identification also benefits from this approach.

Analysis of high transverse momentum \Hbb decays, where the Higgs boson is produced in association with a vector boson, was performed using \intlumi of \SI{13}{\TeV} proton-proton collision data from \runtwo of the \LHC.
This analysis provided first measurements of the \VHbb process in two high transverse momentum regions, and is described with a particular focus on the background modelling studies performed by the author.