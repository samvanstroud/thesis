\chapter{Introduction}\label{chap:intro}


This thesis describes various efforts in improving the understanding of the Higgs boson and its coupling to heavy flavour quarks, primarily through the improvement of the algorithms used to reconstruct and analyse jets.

\cref{chap:theory} describes the theoretical foundations of the work presented in the rest of the thesis.

\cref{chap:lhc_atlas} describes the \ATLAS detector at the \CERN accelerator complex. Details of reconstructed physics objects are also provided.

\cref{chap:tracking} provides an overview of tracking and \btagging at \ATLAS, and studies into the challenges of high transverse momentum \btagging.

\cref{chap:track_classification_mva} describes an tool to predict the origins of tracks. The tool is used to improve \btagging performance by the identification and removal of fake tracks before their input to the \btagging algorithms.

\cref{chap:gnn_tagger} introduces a novel monolithic approach to \btagging using graph neural networks and auxiliary training objectives.

\cref{chap:vhbb_boosted} describes the measurement of the associated production of a Higgs boson decaying into a pair of $b$-quarks at high transverse momentum.

\cref{chap:conclusion} contains some concluding remarks.


\clearpage

The author's contribution to the work presented in this thesis is as follows.

\textbf{Tracking}:
The author was been an active member of the Cluster and Tracking in Dense Environments group for the duration of their qualification task on the understanding of tracking performance at high transverse momentum.
The author played a key role in software r22 validation studies for the tracking group, including the validation of the quasi-stable particle interaction simulation and the radiation damage Monte-Carlo simulation. 
The author helped design and improve several tracking software frameworks, and contributed to heavy flavour tracking efficiency studies in dense environments.

\textbf{$b$-tagging}:
The author has been an active member of the Flavour Tagging group since September 2014. 
The author played a key role in investigating the performance of the low level taggers at high transverse momentum and led studies into the labelling and classification of track origins.
Based on work by J. Shlomi, the author helped develop a new flavour tagging algorithm which offers a large performance improvement with respect to the current state of the art.
The author also played a key role in software r22 validation studies for the Flavour Tagging group, including the validation of the quasi-stable particle interaction simulation.
The author maintains and contributes to various software frameworks used in the Flavour Tagging group, and contributes to group documentation.

\textbf{Higgs}:
The author was an active member of the Boosted VHbb analysis group.
The author performed various studies deriving systematic uncertainties for the \Vjets and diboson backgrounds.
The author also produced and maintained samples, ran fit studies and cross checks, and gave the diboson unblinding approval talk to the Higgs group.
The author also contributed to the developement of the analysis software.
