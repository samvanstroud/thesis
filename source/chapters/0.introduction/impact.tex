% max 500 words

This thesis details research in experimental particle physics.
The primary contributions are on the improvement of the data analysis algorithms which are used to process proton-proton collisions induced within the ATLAS detector at the Large Hadron Collider (LHC), and the analysis of candidate Higgs boson events.

The primary outcome of the research is an advancement of knowledge about how the Universe works on the most fundamental level, encoded for example in the improved measurement of the fundamental constants for the standard model, or in the observation of previously unseen particles or interactions.
Although this kind of knowledge doesn't always have an immediate and direct relevance for society, potential applications are impossible to rule out and could have a very large impact further in the future, as has been seen with previous advancements in fundamental science.

The research does find indirect application in the form of associated technological developements that have transferable application within different fields.
The cutting-edge techniques developed at CERN for ATLAS and the LHC have found many spin-off applications elsewhere in society, for example the World Wide Web, high-field magnet technology in MRI, touch-screen technology and cloud computing.
Fundamental physics, as a proposer of novel and difficult problems, can therefore be seen as a way to generate ideas for new technologies.

Working in the field also helps to train skilled researchers, which can be redeployed to other areas of society to tackle various problems.
In this thesis advanced statistical and data science methods are deployed.
Such methods currently find wide and varied use in many fields.

Finally, the work carried at ATLAS and the LHC is widely publicised -- support of and interest in fundamental physics research helps to generate excitement about science and technology, and educate people about how the Universe works.
This in turn attracts people into the area, propagating the benefits described above.
