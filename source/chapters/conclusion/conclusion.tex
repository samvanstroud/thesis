\chapter{Conclusion}
\label{chap:conclusion}

\section{Summary}\label{sec:conc-summary}

The current understanding of particle physics contains many unanswered questions, and improving our understanding of the Standard Model is a promising way to attempt to answer some of them.
One of the key particles which may enhance this understanding is the Higgs boson, which was first observed only a decade ago and remains under intense scrutiny at the \LHC.
Given it's propensity to decay to heavy flavour \bquarks, reconstructing and identifying \bjets is of crucial importance to improving our understanding in this area.
As discussed in \cref{chap:tracking}, this task becomes increasingly difficult at high transverse momenta.

One of the effects that hampered tracking and \btagging performance at \highpt was identified to be the increased rate of fake tracks.
To address this issue, an algorithm was developed which is able to successfully identify fake tracks within jets \pct{45} of the time, with a minimal loss of signal tracks of \pct{1.2}.
Removal of such tracks was found to improve the \ljet mistagging rate of the SV1 and JetFitter algorithms by up to \pct{20} at high transverse momentum.

A novel approach to \btagging, \GNN was also developed using a Graph Neural Network (GNN) architecture.
The model is encouraged to learn the topology of the jet through vertexing and track classification auxiliary tasks.
As a single end-to-end trained model, \GNN simplifies the complexity of the flavour tagging pipeline and is able to achieve superior performance to the current state-of-the-art algorithms, which rely on a two-tiered approach.
Compared with \DLr, \GNN improves the \lrej by a factor of \ttbllo for \ttbarjets with \ttbarpt at the \WP{70}{b} and by a factor of \zpbllo for \Zprimejets with \Zprimept for a corresponding \bjet efficiency of \pct{30}.
\GNN also demonstrates a significant improvement in the discrimination between $b$- and \cjets, which also contributes to improved \ctagging performance.
In the \highpt regime, \GNN improves the \bjet tagging efficiency by \pct{100} for a fixed \lrej of 100.
Initial validation of the model in data has been performed.
The level of agreement between the simulation and data that is observed is similar to previous flavour tagging algorithms.
\GNN has been successfully deployed in the ATLAS High Level Trigger, and shows promising performance when trained on high pile-up samples corresponding to the HL-LHC conditions.
Ultimately the improved jet tagging performance enabled by the new algorithm will have a large impact across a broad spectrum of the ATLAS physics programme.

This thesis demonstrates that even with suboptimal track reconstruction in this regime, it is possible to make algorithmic advancements to the flavour tagging pipeline to improve the identification of \bjets.
This work has impacts for any analysis which relies on the identification of \bjets, including those which are sensitive to the Higgs boson.

Analysis of \VHbb events was also carried out with \intlumi of \runtwo \ATLAS at \come{13}.
Various background modelling uncertainties were derived and investigations into the fit model were carried out.
The analysis observed a signal strength of 
$\muVH = 0.72 ^{+0.39}_{-0.36} = 0.72 ^{+0.29}_{-0.28} \mathrm{(stat.)} ^{+0.26}_{-0.22} \mathrm{(syst.)}$
corresponding to an observed (expected) significance of $2.1\sigma$ ($2.7\sigma$).
The result was validated using a simultaneous fit to the \VZbb process, which acts as a cross check to validate the primary analysis.
The results of the analysis are the most precise measurements available in the \highpt for the \VHbb process.
The \highpt region is of particular interest as it is a region of phase space with good sensitivity to new physics.


\section{Future Work}\label{sec:conc-future}

Additional algorithmic improvements are likely to yield further improved flavour tagging performance.
Aside from these, large improvements to the flavour tagging performance at \highpt will be possible if the \bhadron decay track reconstruction efficiency and accuracy is improved.
Investigations into such improvements, for example loosening the track reconstruction requirements in \highpt environments, are currently ongoing.

At the moment only the tracks from the Inner Detector and kinematic information about the jet are provided as inputs to the tagging algorithms.
In \cref{chap:gnn_tagger} it was shown that the addition of a simple track-level variable corresponding to the ID of the reconstruction lepton to the model improved tagging performance.
However there is still untapped potential in the form of additional information from the full parameters of the reconstructed leptons (making full use of the Calorimeters and Muon Spectrometer), the calorimeter clusters, and even the individual clusters which are used to reconstruct tracks.
Providing such additional inputs to the model is likely to complement the information provided by the tracks and further improve performance.

Additional auxiliary training objectives may yield improved performance and also help to add to the explainability of the model.
Regression of jet-level quantities such as the transverse momentum and mass, in addition to the truth \bhadron decay length are promising targets.

The \GNN architecture can be easily optimised for new use cases and topologies, as demonstrated by the studies described in \cref{sec:gnn_trig_upgrade}.
Other opportunities include a model with only cluster-based inputs, which could be used for a fast trigger preselection on jets without the need to run the computationally expensive tracking algorithms, or improved primary vertexing and pile-up jet tagging algorithms.

For an improved analysis of the \VHbb process, the following considerations could be taken into account.
Firstly, the addition of \runthree data will provide a significant increase in statistics and a corresponding reduction in statistical uncertainties.
Improved \btagging, enabled by the \GNN model, will also improve the sensitivity of the analysis through improvements in the signal-to-background ratio.
A dedicated $X \rightarrow bb$ version of \GNN which is trained to identify the flavour of \largeR jets directly would also be of benefit.
Improvements in signal-to-background ratio could also be further improved through the use of a dedicated MVA to select signal events, rather than relying on a series of selection cuts.
Finally, the dominant systematic uncertainties relating to the modelling of \largeR jets could be reduced via improved reconstruction and calibration techniques.
