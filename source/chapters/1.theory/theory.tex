\chapter{Theoretical Framework}\label{chap:theory}

\begin{itemize}
  \item Introduce sm
  \item brief history
  \item current areas of study
  \item Reference relevenace to rest of thesis (studying hbb)
\end{itemize}

The Standard Model (SM) of particle physics is the theory describing all known elementary particles and their interactions via three of the four fundamental forces.
Developed by merging the successful theories of classical quantum mechanics and relativity in the second half of the 20th century, the SM's position today at the centre of our understanding of the nature of the universe is firmly established by an unparalleled level of agreement between the predictions from the model and experimental results \cite{morel2020determination,sailer2022measurement}.

The SM has predicted the discovery of the top and bottom quarks \cite{CDF:1995wbb,D0:1995jca,Herb:1977ek}, the \Wboson and \Zboson bosons \cite{UA1:1983crd}, and the tau neutrino \cite{DONUT:2000fbd}.
The last missing piece of the SM to be discovered was the Higgs boson, first posited in \tofill{X}.
After its discovery in 2012 \tofill{citation}, much work has been ongoing on carrying out detailed measurements of its mass and interactions with other particles.

This thesis looks at understanding Higgs decays\dots


\section{The Standard Model}\label{sec:standard_model}

\begin{itemize}
  \item Introduce QFT
  \item Introduce SM Gauge symmetry
  \item List Contents of SM (different particles) masses and charges
  \item Write SM Lagrangian term break up LEW etc
  \item Walk through (or subsection) for each term
\end{itemize}

The SM is formulated in the language of Quantum Field Theory (QFT).
In this framework, particles are localised excitations of corresponding quantum fields, which are operator-valued distribution across spacetime.

Central to QFT is the Lagrangian density which describes the kinematics and dynamics of a field.
%Through observation of conserved quantities, we can derive corresponding symmetries which are expressed in the Lagrangian.
Observations of conserved quantities are linked, via Noether's theorem, to symmetries which are expressed by the Lagrangian.
Alongside Global Poincar\'e symmetry, the SM Lagrangian observes a local non-Abelian $SU(3)_C \otimes SU(2)_L \otimes U(1)_Y$ gauge symmetry.
Gauge symmetries leave observable properties of the system unchanged when certain gauge transformations are applied to the fields.


\begin{table}[!htbp]
  \footnotesize\centering
  \setlength{\tabcolsep}{0.5em} % for the horizontal padding
  \caption{
      The half-integer spin fermions of the SM \cite{Workman:2022ynf}.
      }
  \begin{tabular}{c|ccc|ccc}
      \toprule 
      \multicolumn{1}{c|}{} & \multicolumn{3}{c|}{Leptons} & \multicolumn{3}{c}{Quarks} \\
      \hline
      \textbf{Generation} & \textbf{Flavour} & \textbf{Mass} [\unit\MeV] & \textbf{Charge} [\unit\elementarycharge] & 
                            \textbf{Flavour} & \textbf{Mass} [\unit\MeV] & \textbf{Charge} [\unit\elementarycharge] \\
      \hline
      \multirow{2}{*}{First} & 
        $e$        & $0.511$               & -1 & $u$ & $2.16$ & \nicefrac{2}{3} \\
      & $\nu_e$    & $<1.1 \times 10^{-6}$ &  0 & $d$ & $4.67$ & \nicefrac{-1}{3} \\
      %
      \hline
      \multirow{2}{*}{Second} & 
        $\mu$      & $105.7$ & -1 & $c$ & $1.27 \times 10^{3}$ & \nicefrac{2}{3} \\
      & $\nu_\mu$  & $<0.19$ &  0 & $s$ & $93.4$               & \nicefrac{-1}{3} \\
      %
      \hline
      \multirow{2}{*}{Third} & 
        $\tau$     & $1776.9$& -1 & $t$ & $173 \times 10^{3} $ & \nicefrac{2}{3} \\
      & $\nu_\tau$ & $<18.2$ &  0 & $b$ & $4.18  \times 10^{3} $ & \nicefrac{-1}{3} \\
      \bottomrule
  \end{tabular}
  %\vspace{4mm}
  \label{tab:sm_fermions}
\end{table}


\begin{table}[!htbp]
  \footnotesize\centering
  \setlength{\tabcolsep}{0.5em} % for the horizontal padding
  \caption{
      The integer spin bosons of the SM. The photon, weak bosons and gluons are gauge bosons arising from gauge symmetries \cite{Workman:2022ynf}.
      }
  \begin{tabular}{lcccc}
      \toprule 
      \textbf{Name} & \textbf{Symbol} & \textbf{Mass} [\unit\GeV] & \textbf{Charge} [\unit\elementarycharge] & \textbf{Spin} \\
      \hline
      Photon      & \photon   & $< 1 \times 10^{-27}$     & $< 1 \times 10^{-46}$      & 1    \\
      Weak boson  & \Wpm      & $80.377 \pm 0.012$     & $\pm 1$    & 1    \\
      Weak boson  & \Zboson   & $91.1876 \pm 0.0021$     & 0    & 1    \\
      Gluon       & \gluon    & 0     & 0.5    & 1    \\
      Higgs       & \higgs    & $125.25 \pm  0.17$     & 0    & 0    \\
      \bottomrule
  \end{tabular}
  %\vspace{4mm}
  \label{tab:sm_bosons}
\end{table}


\subsection{Quantum Electrodynamics}\label{sec:qed}

Consider a Dirac spinor field $\psi = \psi(x)$ and its adjoint $\overline{\psi} = \psi^\dagger \gamma^0$, where $\psi^\dagger$ denotes the Hermitian conjugate of $\psi$.
The field $\psi$ describes fermionic \spinhalf particle, for example an electron.
The Dirac Lagrangian density is
%
\begin{equation}\label{eq:dirac_lagrangian}
  \mathcal{L}_{\textnormal{Dirac}} = \overline{\psi} (i \slashed{\partial}  - m )\psi,
\end{equation}
%
where $\slashed{\partial} = \gamma^\mu \partial_\mu$ denotes the contraction with the Dirac gamma matrices $\gamma^\mu$, and summation over up-down pairs of indices is assumed.
Application of the Euler-Lagrange equation on \cref{eq:dirac_lagrangian} yields the Dirac equation
%
\begin{equation}\label{eq:dirac_eq}
  (i \slashed{\partial}  - m )\psi = 0.
\end{equation}
%
Suppose some fundamental symmetry that requires invariance under a $U(1)$ local gauge transformation
%
\begin{equation}\label{eq:U(1)_transformation}
  \psi \rightarrow \psi' = \psi e^{- i q \alpha(x)} ,
\end{equation}
%
where $\alpha$ varies over every spacetime point $x$.
Under this transformation, the Dirac equation transforms as 
%
\begin{equation}\label{eq:dirac_eq_transformed}
  (i \slashed{\partial} - m ) \psi e^{- i q \alpha(x)} + q \slashed{\partial}\alpha(x) \psi e^{- i q \alpha(x)} = 0.
\end{equation}
%
For the Dirac equation to remain invariant under the transformation in \cref{eq:U(1)_transformation}, a new field $A_\mu$ which transforms as $A_\mu \rightarrow A'_\mu = A_\mu + \partial_\mu \alpha(x)$ must be added.
The transformed interaction term
%
\begin{equation}
  - q \slashed{A} \psi \rightarrow - q \slashed{A} \psi e^{- i q \alpha(x)} - q \slashed{\partial} \alpha(x) \psi e^{- i q \alpha(x)}
\end{equation}
%
will then cancel the asymmetric term in \cref{eq:dirac_eq_transformed} as required.
The $U(1)$ invariant Lagrangain can therefore be constructed by adding an interaction between $\psi$ and $A_\mu$ to \cref{eq:dirac_lagrangian}. The kinetic term for the the new field $A_\mu$ is also added in terms of $F_{\mu\nu} = \partial_\mu A_\nu - \partial_\nu A_\mu$, which is trivially invariant under the transformation in \cref{eq:U(1)_transformation}.
The interaction term is absorbed into the covariant derivative $D_\mu = \partial_\mu + i q A_\mu$.
The covariant derivate $D_\mu \psi$ is convenient to work with as it transforms in the same way as the field $\psi$.
This yields the QED Lagrangain
%
\begin{equation}\label{eq:qed_lagrangian}
  \mathcal{L}_{\textnormal{QED}} = -\frac{1}{4} F_{\mu\nu} F^{\mu\nu} + \overline{\psi} (i \slashed{D} - m )\psi,
\end{equation}
%
A quadratic term $A_\mu A^\mu$ is not invariant and therefore the the field $A_\mu$ must be massless.
Requiring invariance under local $U(1)$ gauge transformations necessitated the addition of a new field $A_\mu$, corresponding to photons, which interact with charged matter. \todo{improve interpretation}


\subsection{Quantum Chromodynamics}\label{sec:qcd}

\subsection{The Electroweak Sector}\label{sec:ew_sector}

The $SU(3)_C \otimes SU(2)_L \otimes U(1)_Y$ is spontaneously broken to $SU(3)_C \otimes U(1)_\gamma$.

\section{The Higgs Mechanism}\label{sec:sm_higgs}

\begin{itemize}
  \item Motivation
  \item Walkthrough
\end{itemize}

\subsection{Electroweak Symmetry Breaking}\label{sec:ew_symmetry_breaking}
\subsection{Fermionic Yukawa Coupling}\label{sec:higgs_yukawa_coupling}

