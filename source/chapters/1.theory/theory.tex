\chapter{Theoretical Framework}\label{chap:theory}

\begin{itemize}
  \item Introduce sm
  \item brief history
  \item current areas of study
  \item Reference relevenace to rest of thesis (studying hbb)
\end{itemize}

The Standard Model (SM) of particle physics is the theory describing all known elementary particles and their interactions via three of the four fundamental forces.
Developed by merging the successful theories of classical quantum mechanics and relativity in the second half of the 20th century, the SM's position today at the centre of our understanding of the nature of the universe is firmly established by an unparalleled level of agreement between the predictions from the model and experimental results \cite{morel2020determination,sailer2022measurement}.

The SM has predicted the discovery of the top and bottom quarks \cite{CDF:1995wbb,D0:1995jca,Herb:1977ek}, the \Wboson and \Zboson bosons \cite{UA1:1983crd}, and the tau neutrino \cite{DONUT:2000fbd}.
The last missing piece of the SM to be discovered was the Higgs boson, first posited in \tofill{X}.
After its discovery in 2012 \tofill{citation}, much work has been ongoing on carrying out detailed measurements of its mass and interactions with other particles.

This thesis looks at understanding Higgs decays\dots


\section{The Standard Model}\label{sec:standard_model}

The SM is formulated in the language of Quantum Field Theory (QFT).
In this framework, particles are localised excitations of corresponding quantum fields, which are operator-valued distribution across spacetime.

Central to QFT is the Lagrangian density which describes the kinematics and dynamics of a field.
Observations of conserved quantities are linked, via Noether's theorem, to symmetries which are expressed by the Lagrangian.
Alongside Global Poincar\'e symmetry, the SM Lagrangian observes a local non-Abelian $SU(3)_C \otimes SU(2)_L \otimes U(1)_Y$ gauge symmetry.
Gauge symmetries leave observable properties of the system unchanged when certain gauge transformations are applied to the fields.
The full Lagrangain of the SM can be broken up into distinct terms corresponding to the different sectors.
%
\begin{equation}\label{eq:sm_lagrangian}
  \mathcal{L}_{\textnormal{SM}} = \mathcal{L}_{\textnormal{EW}} + \mathcal{L}_{\textnormal{QCD}} + \mathcal{L}_{\textnormal{Higgs}} + \mathcal{L}_{\textnormal{Yukawa}}
\end{equation}
%
The SM provides a mathematical description of how the four fundamental forces interact with the matter content of the universe.
The SM contains $12$ \spinhalf fermions, listed in \cref{tab:sm_fermions}, and $5$ bosons listed in \cref{tab:sm_bosons}.
%
\begin{table}[!htbp]
  \footnotesize\centering
  \setlength{\tabcolsep}{0.5em} % for the horizontal padding
  \begin{tabular}{c|ccc|ccc}
      \toprule 
      \multicolumn{1}{c|}{} & \multicolumn{3}{c|}{Leptons} & \multicolumn{3}{c}{Quarks} \\
      \hline
      \textbf{Generation} & \textbf{Flavour} & \textbf{Mass} [\unit\MeV] & \textbf{Charge} [\unit\elementarycharge] & 
                            \textbf{Flavour} & \textbf{Mass} [\unit\MeV] & \textbf{Charge} [\unit\elementarycharge] \\
      \hline
      \multirow{2}{*}{First} & 
        $e$        & $0.511$               & -1 & $u$ & $2.16$ & \nicefrac{2}{3} \\
      & $\nu_e$    & $<1.1 \times 10^{-6}$ &  0 & $d$ & $4.67$ & \nicefrac{-1}{3} \\
      %
      \hline
      \multirow{2}{*}{Second} & 
        $\mu$      & $105.7$ & -1 & $c$ & $1.27 \times 10^{3}$ & \nicefrac{2}{3} \\
      & $\nu_\mu$  & $<0.19$ &  0 & $s$ & $93.4$               & \nicefrac{-1}{3} \\
      %
      \hline
      \multirow{2}{*}{Third} & 
        $\tau$     & $1776.9$& -1 & $t$ & $173 \times 10^{3} $ & \nicefrac{2}{3} \\
      & $\nu_\tau$ & $<18.2$ &  0 & $b$ & $4.18  \times 10^{3} $ & \nicefrac{-1}{3} \\
      \bottomrule
  \end{tabular}
  \caption{
    The half-integer spin fermions of the SM \cite{Workman:2022ynf}.
    Three generations of particles are present.
    Also present (unlisted) are the antiparticles, which are identical to the particles up to a reversed charge sign.
    }
  \label{tab:sm_fermions}
\end{table}
%

%
\begin{table}[!htbp]
  \footnotesize\centering
  \setlength{\tabcolsep}{0.5em} % for the horizontal padding
  \begin{tabular}{lcccc}
      \toprule 
      \textbf{Name} & \textbf{Symbol} & \textbf{Mass} [\unit\GeV] & \textbf{Charge} [\unit\elementarycharge] & \textbf{Spin} \\
      \hline
      Photon      & \photon   & $< 1 \times 10^{-27}$     & $< 1 \times 10^{-46}$      & 1    \\
      Weak boson  & \Wpm      & $80.377 \pm 0.012$     & $\pm 1$    & 1    \\
      Weak boson  & \Zboson   & $91.1876 \pm 0.0021$     & 0    & 1    \\
      Gluon       & \gluon    & 0     & 0.5    & 1    \\
      Higgs       & \higgs    & $125.25 \pm  0.17$     & 0    & 0    \\
      \bottomrule
  \end{tabular}
  \caption{
    The integer spin bosons of the SM \cite{Workman:2022ynf}.
    The photon, weak bosons and gluons are gauge bosons arising from gauge symmetries, and carry the four fundamental forces of the SM.
  }
  \label{tab:sm_bosons}
\end{table}
%
\begin{comment}
  In quantum physics and chemistry, quantum numbers describe values of conserved quantities in the dynamics of a quantum system. Quantum numbers correspond to eigenvalues of operators that commute with the Hamiltonian—quantities that can be known with precision at the same time as the system's energy[note 1]—and their corresponding eigenspaces. Together, a specification of all of the quantum numbers of a quantum system fully characterize a basis state of the system, and can in principle be measured together.
\end{comment}


\subsection{Quantum Electrodynamics}\label{sec:qed}

Consider a Dirac spinor field $\psi = \psi(x)$ and its adjoint $\overline{\psi} = \psi^\dagger \gamma^0$, where $\psi^\dagger$ denotes the Hermitian conjugate of $\psi$.
The field $\psi$ describes fermionic \spinhalf particle, for example an electron.
The Dirac Lagrangian density is
%
\begin{equation}\label{eq:dirac_lagrangian}
  \mathcal{L}_{\textnormal{Dirac}} = \overline{\psi} (i \slashed{\partial}  - m )\psi,
\end{equation}
%
where $\slashed{\partial} = \gamma^\mu \partial_\mu$ denotes the contraction with the Dirac gamma matrices $\gamma^\mu$ (summation over up-down pairs of indices is assumed).
Application of the Euler-Lagrange equation on \cref{eq:dirac_lagrangian} yields the Dirac equation
%
\begin{equation}\label{eq:dirac_eq}
  (i \slashed{\partial}  - m )\psi = 0.
\end{equation}
%
Suppose some fundamental symmetry that requires invariance under a $U(1)$ local gauge transformation
%
\begin{equation}\label{eq:U(1)_transformation}
  \psi \rightarrow \psi' = \psi e^{- i q \alpha(x)} ,
\end{equation}
%
where $\alpha$ varies over every spacetime point $x$.
Under this transformation, the Dirac equation transforms as 
%
\begin{equation}\label{eq:dirac_eq_transformed}
  (i \slashed{\partial} - m ) \psi e^{- i q \alpha(x)} + q \slashed{\partial}\alpha(x) \psi e^{- i q \alpha(x)} = 0.
\end{equation}
%
For the Dirac equation to remain invariant under the transformation in \cref{eq:U(1)_transformation}, a new field $A_\mu$ which transforms as $A_\mu \rightarrow A'_\mu = A_\mu + \partial_\mu \alpha(x)$ must be added.
The transformed interaction term
%
\begin{equation}
  - q \slashed{A} \psi \rightarrow - q \slashed{A} \psi e^{- i q \alpha(x)} - q \slashed{\partial} \alpha(x) \psi e^{- i q \alpha(x)}
\end{equation}
%
will then cancel the asymmetric term in \cref{eq:dirac_eq_transformed} as required.
The $U(1)$ invariant Lagrangain can therefore be constructed by adding an interaction between $\psi$ and $A_\mu$ to \cref{eq:dirac_lagrangian}. The kinetic term for the the new field $A_\mu$ is also added in terms of $F_{\mu\nu} = \partial_\mu A_\nu - \partial_\nu A_\mu$, which is trivially invariant under the transformation in \cref{eq:U(1)_transformation}.
The interaction term is absorbed into the covariant derivative $D_\mu = \partial_\mu + i q A_\mu$, thus named as it transforms in the same way as the field $\psi$.
This yields the QED Lagrangain
%
\begin{equation}\label{eq:qed_lagrangian}
  \mathcal{L}_{\textnormal{QED}} = -\frac{1}{4} F_{\mu\nu} F^{\mu\nu} + \overline{\psi} (i \slashed{D} - m )\psi ,
\end{equation}
%
A quadratic term $A_\mu A^\mu$ is not invariant and therefore the the field $A_\mu$ must be massless.
Requiring invariance under local $U(1)$ gauge transformations necessitated the addition of a new field $A_\mu$, corresponding to photons, which interact with charged matter. \todo{improve interpretation}


\subsection{Quantum Chromodynamics}\label{sec:qcd}

Quantum Chromodynamics (QCD) is the study of quarks, gluons and their interactions.
Quarks and gluons carry colour charge, which comes in three kinds, called red, green and blue.
While the $U(1)$ symmetry group in \cref{sec:qed} was Abelian, the QCD Lagrangian is specified by requiring invariance under transformations from the non-Abelian $SU(3)$ group, making it a Yang\nobreakdash-Mills theory \cite{PhysRev.96.191} which requires the addition of self-interacting gauge fields.
The infinitesimal $SU(3)$ group generators are given by $T_a = \lambda_a / 2$, where $\lambda_a$ are the eight Gell\nobreakdash-Mann matrices.
These span the space of infinitesimal group transformations and do not commute with each other, instead satisfying the commutation relation
%
\begin{equation}
  \com{T_a}{T_b} = i f_{abc} T_c ,
\end{equation}
%
where $f_{abc}$ are the group's structure constants.
Consider the six quark fields $q_k = q_k(x)$.
Each flavour of quark $q_k$ transforms in the fundamental triplet representation, where each component of the triplet corresponds to a different value of the colour quantum number.
$G^{a}_{\mu\nu}$ are the eight gluon field strength tensors, one for each generator $T_a$, defined as
%
\begin{equation}\label{eq:qcd_field_strength_tensor}
  G^a_{\mu\nu} = \partial_\mu A_\nu - \partial_\nu A_\mu - g_s f^{abc} A_\mu^b A_\nu^c ,
\end{equation}
%
where $A_\mu^a$ are the gluon fields and $g_s$ is the strong coupling constant. The covariant derivative is written as
%
\begin{equation}\label{eq:qcd_covariant_derivative}
  D_\mu = \partial_\mu + i g_s T_a A_\mu^a .
\end{equation}
%
The full QCD Lagrangain is then given by
%
\begin{equation}\label{eq:qcd_lagrangian}
  \mathcal{L}_\textnormal{QCD} = 
  - \frac{1}{4} G^{a}_{\mu\nu} G_{a}^{\mu\nu}
  + q_k (i \slashed{D} - m_k) q_k .
\end{equation}
%
Cubic and quartic terms of the gauge fields $A^a_\mu$ appear in the Lagrangian, leading to the gluon's self interaction.

The QCD coupling constant $g_s$ varies, or ``runs'', with energy.
At lower energy scales (and corresponding larger distance scales) the interaction is strong.
This leads to quark confinement, whereby an attempt to isolate individual colour-charged quarks requires so much energy that additional quark-antiquark are produced.
At higher energy scales (and corresponding smaller distance scales), asymptotic freedom occurs as the interactions become weaker, allowing perturbative calculations to be performned.
Hadrons are bound states of quarks.
They are invariant under $SU(3)$ gauge transformations (i.e. are colour-charge neutral, or colourless).


\subsection{The Electroweak Sector}\label{sec:ew_sector}

The weak and electromagnetic forces are unified in the Glashow-Weinberg-Salam (GWS) model of electroweak interaction \cite{Glashow:1961tr,Weinberg:1967tq,Salam:1968rm}.
The Lagrangain is specified by requiring invariance under the symmetry group $SU(2)_L \otimes U(1)_Y$, as motivated by a large amount of experimental data.
Here, $SU(2)_L$ is referred to as weak isospin and $U(1)_Y$ as weak hypercharge. (the product of the isospin and hypercharge groups.).

The generators of $SU(2)_L$ are $T^a = \sigma_a/2$, where $\sigma_a$ are the three Pauli spin matrices which satisfy the commutation relation 
%
\begin{equation}
\com{T_a}{T_b} = i \varepsilon_{abc} T_c .
\end{equation}
%

The generator of $U(1)_Y$ is $Y = 1/2$.
Each generator corresponds to an gauge field, which, after symmetry breaking (discussed in \cref{sec:ew_symmetry_breaking}), give rise to the massive vector bosons \Wpm and \Zboson and the massless photon.


\section{The Higgs Mechanism}\label{sec:sm_higgs}

\begin{itemize}
  \item Motivation
  \item Walkthrough
\end{itemize}

\subsection{Electroweak Symmetry Breaking}\label{sec:ew_symmetry_breaking}
\subsection{Fermionic Yukawa Coupling}\label{sec:higgs_yukawa_coupling}

