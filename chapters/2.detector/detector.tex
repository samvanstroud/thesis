\chapter{The Large Hadron Collider and the \ATLAS Detector}
\label{chap:detector}

\section{Overview}
The Large Hadron Collider (\LHC) at \CERN has extended the frontiers of particle physics through its unprecedented energy and luminosity.
In 2010, the \LHC collided proton bunches, each containing more than $10^{11}$ particles, 20 million times per second, providing \unit{7}{\TeV} proton-proton collisions at instantaneous luminosities of up to \peakLumi.

\section{Trigger system}
\label{sec:bg-theory:triggers}
An LHCb trigger table borrowed from \texttt{hepthesis} is shown in \TableRef{tab:bg-theory:trigger_details}:

\begin{table}[bht]
  \begin{tabular}{lllll}
                & L0              & L1              & HLT             \\
    \midrule
    Input rate  & \unit{40}{\MHz} & \unit{1}{\MHz}  & \unit{40}{\kHz} \\
    Output rate & \unit{1}{\MHz}  & \unit{40}{\kHz} & \unit{2}{\kHz}  \\
    Location    & On detector     & Counting room   & Counting room   \\
  \end{tabular}
  \caption{Characteristics of the trigger levels and offline analysis.}
  \label{tab:bg-theory:trigger_details}
\end{table}

\section{Physics Objects}

\subsection{Tracks}
\subsection{Jets}